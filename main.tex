
\documentclass[UTF8,a4paper,zihao=-4]{ctexart}

% ==== 页面与排版(参照硕士论文常见版式做了接近的设置,可按学校模板再微调)====
\usepackage[left=2.6cm,right=2.6cm,top=2.1cm,bottom=2.1cm]{geometry}
\usepackage{setspace}
\setstretch{1.5}
\setlength{\parindent}{2em}
\usepackage{fancyhdr}
\pagestyle{fancy}
\fancyhf{}
\cfoot{\thepage}
\renewcommand{\headrulewidth}{0pt}

% ==== 参考文献:脚注+文末,按引用顺序排序 ====
\usepackage[backend=biber,style=numeric,sorting=none]{biblatex}

\addbibresource{refs.bib}

% 让 \footfullcite 输出“原样参考文献字符串”(来自 Word 参考文献列表)
\DeclareBibliographyDriver{misc}{%
  \printfield{note}%
  \finentry
}
% biblatex:数字编号、按出现顺序
\usepackage[backend=biber,style=numeric,sorting=none]{biblatex}
\addbibresource{refs.bib}

% 让 fullcite 在脚注/文末都“原样输出”(如果你之前用了 note 字段方案就保留)
\DeclareBibliographyDriver{misc}{%
  \printfield{note}\finentry
}

% 一个“角标[1] + 页脚全文献(无重复标记)”的引用命令
\newcommand{\fcite}[1]{%
  % 右上角小角标 [n]
  \textsuperscript{[\citefield{#1}{labelnumber}]}%
  % 页脚输出同号脚注(不会再在正文生成一个脚注标记)
  \footnotetext[\citefield{#1}{labelnumber}]{\fullcite{#1}}%
}

\begin{document}

% ====== 正文 ======
\begin{center}
{\zihao{-2}\bfseries “内核移植”与“外壳重塑”:基于全球本土化视角的国产短剧出海传播策略研究\par}
\vspace{0.6em}
{\zihao{-3}——以ReelShort为例\par}
\end{center}

\begin{abstract}
在数字技术重塑全球传播生态的背景下,中国微短剧出海正面临从内容分发向模式输出的转型。本研究以ReelShort平台为个案,基于全球本土化理论,构建“内核移植”与“外壳重塑”的二元分析框架,解析其跨文化传播策略。\par
研究发现,ReelShort通过“外壳重塑”实施了激进的去语境化生产,利用本土化的视听符号与拟像空间构建,有效消解了异质文化的认知壁垒。同时,其通过“内核移植”保留了中国通俗文艺中历经市场验证的情感结构与伦理逻辑,并借由算法将其转译为具有普适性的生理性共鸣,从而在虚拟社群中实现了情感能量的循环。\par
本研究认为,该模式本质上是作为内核的中国生产经验与作为外壳的全球工业标准的一次深度耦合。这一范式虽面临“文化空心化”等伦理困境,但为中国数字文化产品从“借船出海”迈向“造船出海”提供了重要的战略参照。\par
\end{abstract}
\noindent\textbf{关键词:}ReelShort;微短剧出海;全球本土化;内核移植;外壳重塑;跨文化传播\par
\newpage



\section{研究背景与问题提出}
在数字技术重塑全球传播生态的当下,文化产品的跨国流动正经历着从单向度的“内容输出”向深层次的“模式嵌入”转型的历史性变革。随着移动互联网的普及与受众碎片化消费习惯的固化,以“微短剧”为代表的新型网络视听形态异军突起,成为继网络文学、网络游戏之后,中国数字文化产业“走出去”的新兴增长极 。然而,长期以来,中国影视产品的国际传播往往受制于霍斯金斯所言的“文化折扣”,\fcite{ref01}高语境的文化符号与独特的叙事伦理难以在异质文化土壤中实现无损解码,导致“走出去”容易,“走进去”难。\par
在此背景下,以ReelShort为代表的中国微短剧平台在欧美市场的现象级“破圈”显得尤为引人深思。这一平台并未沿袭传统影视出海的“译制”老路,也未完全照搬好莱坞的叙事范式,而是开创了一种极具张力的“混血”传播策略:一方面,它大胆启用了欧美本土演员、还原了西式生活场景,在视听表层实现了彻底的“去语境化”;另一方面,它却顽强地保留了中国网络文学中历经市场验证的“爽感”机制、情感结构与叙事公式(如“霸总”、“逆袭”、“先婚后爱”),实现了深层叙事逻辑的跨文化“移植”。\par
这种“中国芯、西洋皮”的独特构造,恰如其分地呼应了罗兰·罗伯特森提出的“全球本土化”理论,即“普遍主义的特殊化”与“特殊主义的普遍化”的辩证统一\footfullcite{ref02}。ReelShort的实践表明,在智能传播时代,跨文化传播的核心已不再是单纯的意义传递,而是一种基于算法技术与情感治理的“格式改编”。它通过将具有普世穿透力的情感内核从特定文化语境中剥离,并注入到符合目标市场审美习惯的本土化外壳之中,成功消解了文化认知壁垒,引发了跨越国界的生理性共鸣与情感能量循环 。\par
基于此,本研究试图引入“内核移植”与“外壳重塑”这一二元分析框架,以ReelShort为典型个案,深入剖析这种基于全球本土化视角的出海传播策略及其内在机理。这不仅是对微短剧这一新兴媒介形态的学理审视,更是对中国数字文化产品如何从“借船出海”迈向“造船出海”,进而实现从“内容分发”到“模式输出”的范式升级的一次深度探索。\par
\newpage
\section{核心概念界定}
本研究主要涉及三个关键概念的学理阐释与操作化定义:出海网络微短剧、全球本土化,以及作为本研究核心分析框架的“内核移植”与“外壳重塑”。\par
(一)、出海网络微短剧\par
根据国家广播电视总局的定义,网络微短剧是指单集时长在几分钟以内、有着相对明确的主题和主线、故事情节较为连续和完整的网络影视作品 。本研究中的“出海网络微短剧”,特指国内制作机构或平台针对海外市场生产,通过移动端应用程序(App)进行传播的竖屏微短剧。\par
不同于早期腾讯WeTV、爱奇艺国际版等长视频平台在YouTube上分发的横屏短剧(即“罐装节目”),本研究聚焦的ReelShort等平台所推出的微短剧,具有显著的“社交媒体思维”与“碎片化叙事”特征 。其单集时长通常控制在1-2分钟,采用竖屏画幅以适配移动端用户的观看习惯,通过高密度的剧情反转与悬念设置来争夺用户的碎片化注意力 。这类剧集在产业逻辑上体现为“技术—商业—内容”的完整模式出海,而非单纯的内容译制 。\par
(二)、全球本土化\par
“跨文化传播”的核心议题在于如何消解异质文化间的认知壁垒与“文化折扣”。在数字内容全球化流动的当下,传统的跨文化传播研究已从单纯的“差异比较”转向了动态的“融合适应”。罗兰·罗伯特森提出的“全球本土化”理论,作为跨文化传播在当代媒介研究中的重要理论延伸,精准地揭示了全球性普遍主义(如爽感机制)与地方性特殊主义(如欧美视听符号)之间的辩证统一关系。因此,本研究选取全球本土化作为切入视角,本质上是为了更深入地探究中国微短剧在跨文化传播实践中的具体策略与生成机理。”\par
“全球本土化”一词最早由罗兰·罗伯特森引入社会学领域,意指“普遍主义的特殊化”与“特殊主义的普遍化”的辩证统一过程 。在跨文化传播视域下,它超越了早期的“媒介帝国主义”单向流动论,强调全球文化与地方文化在互动中产生的“混合化”。在本研究中,全球本土化不仅是宏观的理论视角,更是具体的制播策略。它指涉中国微短剧平台在进入欧美市场时,并非简单地进行语言翻译,而是对媒介产品进行“格式改编”。\footfullcite{ref03}即在保留中国微短剧成熟的商业模式与生产机制的同时,针对目标市场的文化语境进行深度的本土适应性改造,以消解跨文化传播中的“文化折扣” 。\par
(三)、“内核移植”与“外壳重塑”\par
基于上述理论,本研究构建了“内核移植”与“外壳重塑”这一分析框架,用以解构ReelShort等平台的出海策略。\par
“内核移植” 指的是保留中国网络文学与微短剧在长期发展中形成的、已被市场验证的深层叙事逻辑与情感结构。研究表明,中国网文中的“爽感”机制(如逆袭、复仇、霸总宠爱)具有跨文化的普适性,能够激发全人类共通的情感共鸣 。这种“本土模式”的输出,构成了出海微短剧的叙事基因,使其在不同文化圈层中仍能维持高强度的用户粘性 。\par
“外壳重塑” 指的是对微短剧表层视听符号进行的彻底本土化编码。为了适应欧美低语境文化的消费习惯,制作方启用了海外本土演员,搭建符合欧美审美的场景,并将中国特有的文化符号(如龙王、修仙)置换为西方受众熟悉的文化符号(如狼人、吸血鬼)。这种策略不同于传统的“鹦鹉模式”,而是类似于“蝴蝶模式”的深度重构。\footfullcite{ref04}(所谓“鹦鹉模式”,即传统的“语言转译”策略,其本质是文化产品的直接搬运,往往因显著的地缘文化差异而遭遇“文化折扣”。 而“蝴蝶模式”则代表了一种进阶的“文化转码”策略。正如毛毛虫通过变态发育重塑为蝴蝶以适应飞行环境,ReelShort 通过对视听符号的彻底置换,使中国故事内核获得了符合西方主流审美的新躯壳,从而在保留情感“爽感”的同时,消除了文化接受的心理阻滞。),通过视听语言的去陌生化,实现中国故事在海外市场的无感落地 。\par
\newpage
\section{文献综述}
在数字内容全球化流动的当下,文化产品的跨国传播早已超越了简单的“输出”与“输入”二元对立,转向了一种更为复杂的动态适应与重构过程。以ReelShort为代表的中国微短剧在欧美市场的“破圈”并非偶然,它是全球媒介技术迭代、跨文化传播范式转型以及中国通俗文艺(如网络文学)长期积累共同作用的结果。为了厘清“内核移植”与“外壳重塑”这一跨文化传播策略的内在机理,本研究基于全球本土化的理论视域,按照“理论层—对象层—策略层”的逻辑层次,对现有文献进行梳理与评述。\par
(一)、理论层:跨文化传播理论视域下的全球本土化转向\par
\subsubsection{跨文化传播研究的范式演进:从“冲突”到“融合”}
跨文化传播作为传播学的重要分支,其核心议题在于探讨不同文化背景的社会成员之间如何进行有效的信息交互与意义共享 。早期的研究多受“媒介帝国主义”理论影响,侧重于分析强势文化对弱势文化的单向渗透,或基于爱德华·霍尔的高/低语境理论探讨文化差异带来的误读与冲突 。\par
然而,随着全球化深度的拓展,简单的二元对立视角逐渐显露出解释力的匮乏。学界开始从强调“文化冲突”转向关注“文化适应”与“融合”。丁梦颖(2025)在梳理学科脉络时指出,当前的跨文化传播已进入转型与反思阶段,不仅关注信息的跨国流动,更关注文化要素在全球社会迁移中的变动与重构过程 。\footfullcite{ref04}在这种背景下,罗兰·罗伯特森提出的“全球本土化理论“应运而生,成为理解当下跨文化传播新范式的核心框架。该理论强调,有效的跨文化传播并非单向的同质化,而是“普遍主义的特殊化”与“特殊主义的普遍化”的辩证统一。\par
跨文化适应策略:消解“文化折扣”的机制\par
在具体的跨文化适应机制上,降低“文化折扣”是传播策略的核心目标。佟亚云与金鑫(2025)基于贝里的文化适应模型指出,微短剧出海实际上采取了“分隔”、“同化”与“融合”的三重跨文化策略。\footfullcite{ref05}其中,“同化”策略(采用本土视听符号)与“融合”策略(输出中华文化精神)的辩证使用,旨在通过文化符号的置换来降低异质文化理解的门槛,从而达成跨文化传播的有效性。\par
\subsubsection{媒介流动的“去中心化”与“平台权力”}
在智能传播时代,文化流动的逻辑正在发生深刻的技术性转向。窦书棋与赵永华(2024)指出,在全球语境下,地方文化必须通过“文化间性”的转换,在尊重异质文化“他者”的基础上,通过观念、语境与情感的转换来实现共存 。\footfullcite{ref06}然而,付超与贾瑞凯(2026)提出了更为批判性的观点,认为ReelShort的成功标志着中国数字内容出海从单纯的“意义传递”转向了“平台权力输出”。\footfullcite{ref07}这种权力不再体现为某种意识形态的说教,而是体现为一种基于算法的“情感治理术”——即利用算法对全球受众的情感需求进行精准分类、规训与满足 。这表明,全球本土化的实践已经从文化的协商上升到了技术与情感的博弈。\par
(二)、对象层:碎片化时代的微短剧与其跨国流动\par
\subsubsection{媒介形态:碎片化叙事与“替代性满足”}
作为研究对象的“微短剧”,被界定为单集时长短、叙事密度高、情节连续的新型网络视听形态。龙小农与李芙蓉(2024)通过对比Netflix与ReelShort指出,不同于Netflix追求的“深度沉浸”与“全球视角”,ReelShort开创了一种“即时娱乐”与“竖屏浅焦”的新范式。\footfullcite{ref08}其核心功能在于为受众提供“替代性满足”,即通过高强度的爽感刺激来填补现实生活的匮乏。从产业演进的角度看,李庚与吴倩含(2025)指出,中国微短剧出海正经历从早期的“流量依赖”向“生态构建”转型。\footfullcite{ref09}早期的爆款逻辑虽然能快速打开市场,但容易导致供需失衡,未来的方向在于“双轨布局”——即版权分销与本土原创并重,并依托技术赋能实现产业链的延伸 。\par
\subsubsection{情感流动:互动仪式链与虚拟共在}
这种媒介形态的成功,还离不开独特的社群互动机制。庞华等(2025)引入“互动仪式链”理论,发现ReelShort虽然App内部封闭,但通过YouTube等社交媒体的评论区构建了“虚拟共在”场域。\footfullcite{ref10}受众在“吐槽”与“沉迷”的矛盾心理中积累“情感能量”,并将其转化为跨平台迁移的动力 。这表明,微短剧的传播不仅是内容的流动,更是情感能量在数字社群中的循环与增值。\par
(三)、 策略层:“内核”与“外壳”的二元互动研究\par
\subsubsection{“外壳重塑”:去语境化生产与视听符号的同化}
为了适应欧美低语境文化,微短剧在表层符号上进行了彻底的重塑,这构成了本研究“外壳重塑”的核心内涵。胡颢琛(2025)通过对比YouTube上的横屏与竖屏短剧发现,竖屏微短剧(如ReelShort)采取了彻底的“全球本土化”策略,即对影视文本进行“格式改编”,而非仅仅添加字幕的“罐装出口” 。\footfullcite{ref11}王沛楠(2024)将其描述为“本土模式,海外内容”,即利用欧美本土演员、场景和语言,包装中国式的生产模式 。\footfullcite{ref12}付超与贾瑞凯(2026)进一步将其提炼为“去语境化生产”。\footfullcite{ref07}这一策略包括物理空间的“拟像化”(刻意剥离具体的城市地标)、表演的“错位操演”(欧美演员执行中国式夸张表演)以及接受情境的“原子化” 。通过这种视觉上的“去陌生化”,ReelShort消除了西方受众对异质文化的排斥感,构建了一个全球通用的算法背景。\par
\subsubsection{“内核移植”:情感结构的算法转译与模因复制}
在剥离了中式外壳后,保留下来的是什么?这正是“内核移植”所要探讨的问题。戴瑶琴与张晖敏(2024)指出,微短剧的内核是“爽点”生产机制与伦理内核。\footfullcite{ref13}中国网文中的“霸总”、“逆袭”、“复仇”等母题,实际上对应了全球共通的爱欲想象与“回报主义”伦理期待(善恶有报) 。ReelShort通过“标签化”生产(如将“龙王”替换为“狼人/吸血鬼”),实现了文化模因的跨国复制。更为深刻的是,付超与贾瑞凯(2026)提出了“算法情感转译”概念。\footfullcite{ref07}他们认为,ReelShort输出的并非具象的中国故事,而是将中国网文中的“抑—扬”结构通过算法转化为具有跨文化普适性的“生理性共鸣”。这种机制对应着人类大脑的多巴胺奖赏机制,成功将东亚宗族伦理(如面子、赘婿)转译为西方语境下的阶级逻辑与权力关系(如Underdog逆袭)。丁梦颖(2025)也强调,这种基于人类共通欲望的叙事逻辑,构成了中国通俗文化出海的坚实“内核” 。\footfullcite{ref04}\par
(四)、文献述评与研究空间\par
综上所述,既有研究已构建了较为立体的解释框架:宏观上是全球本土化与文化适应理论,中观上是平台模式与媒介生态位分析,微观上是“爽点”模因与情感转译机制的探讨。学者们普遍认可,ReelShort的成功在于其精准地运用了全球本土化策略,通过“技术—商业—内容”的模式出海降低了文化折扣。然而,审视既有文献,仍存在值得进一步深耕的空间: 第一,现有研究多侧重于对“外壳”(本土化制作)或“内核”(爽感机制)的独立分析,缺乏构建一个整合性的“核—壳”二元互动模型,去系统阐释外壳的“去语境化”是如何具体服务于内核的“无损传输”的。第二,虽然付超等学者指出了“情感转译”的重要性,但对于这种“中国芯、西洋皮”的混合文本在受众心理层面的动态接纳机制(如从认知冲突到情感沉浸的转化过程)仍缺乏细致的微观剖析。\par
基于此,本研究拟引入“内核移植”与“外壳重塑”的分析框架,以ReelShort为典型个案,深入剖析这种基于“全球本土化”视角的出海传播策略,旨在揭示中国数字文化产品如何通过“形式的西化”实现“逻辑的输出”。\par
\newpage
\section{研究设计}
(一)、研究对象选取\par
本研究遵循目的性抽样原则,选取ReelShort平台及其头部爆款短剧作为核心样本。这一选择并非随机,而是基于该平台及文本在微短剧出海版图中的典型性与代表性考量。\par
\subsubsection{平台选取的典型性}
ReelShort 系中国数字出版领军企业“中文在线”旗下子公司枫叶互动所孵化的微短剧应用。作为当前中国微短剧出海领域商业化最成功、市场渗透率最高的平台,其在2024年多次登顶美区iOS娱乐榜单,甚至一度超越TikTok 。ReelShort 的崛起具有里程碑意义,它标志着中国影视出海已从早期的单纯“内容译制”通过简单的语言转换输出,转向了深度的“本土化自制”与产业链级别的“模式输出” 。选取该平台作为个案,能够最大程度地透视中国数字文化产品在欧美主流市场的生存逻辑。\par
文本选取的代表性\par
为确保分析样本能够覆盖不同的出海策略维度,本研究选取了ReelShort平台上两部具有范式意义的爆款剧集作为具体剖析对象:\par
《Fated to My Forbidden Alpha》(命中注定的禁忌阿尔法): 该剧是“题材置换型”文本的典范。它成功地将中国网文逻辑中关于“庶女逆袭”与“血统觉醒”的叙事内核,嫁接于西方受众熟知的“狼人(Werewolf)”题材外壳之上,集中体现了“同化策略”在跨文化传播中的效用 。\par
《The Double Life of My Billionaire Husband》(亿万富翁老公的双面生活): 该剧则是“现代都市型”文本的代表。尽管其保留了中国经典的“赘婿”、“霸总”及“先婚后爱”等叙事逻辑,但在视听符号层面进行了彻底的欧美本土化重构,展现了另一种维度的文化适应路径 。\par
(二)、研究方法\par
本研究遵循“文本—受众—策略”的逻辑进路,综合运用以下三种研究方法,以实现从微观文本到宏观机制的立体化阐释:\par
\subsubsection{文本分析法}
以符码理论为基础,对选定的目标短剧展开由表及里的深度剖析。在表层视听维度,本研究聚焦于场景搭建、演员选角及服化道设计,分析其如何利用“去语境化生产”策略剥离原有的文化地缘特征,从而构建出契合目标市场审美的拟像空间;在深层叙事维度,研究进一步拆解“黄金三秒”原则、悬念钩子设置及反转节奏,探究那些能够引发受众生理性快感的“爽感机制”,是如何在算法逻辑的规训下被标准化与再生产的。\par
\subsubsection{网络民族志}
为了突破单纯的文本封闭性,探究受众的真实解码过程,本研究将田野场域延伸至ReelShort在TikTok的官方账号评论区。首先抓取上述两部剧集相关视频下的高赞评论作为分析语料。然后引入互动仪式链理论,通过观察受众的情感表达(如吐槽、催更、沉迷声明),分析分散的个体受众如何通过“虚拟共在”形成情感能量的循环与增值,并考察其中是否存在跨文化的“对抗式解码”或协商过程 。\par
案例分析法\par
将ReelShort置于全球媒介产业变革的宏观背景下进行审视。结合产业转型路径理论,分析该平台如何从早期的“流量依赖”模式向“生态构建”模式演进 ,并探究其如何通过“技术—商业—内容”的组合拳策略,突破文化折扣的壁垒,实现跨文化传播的突围 。\par
(三)、分析框架:基于全球本土化的“核-壳”二元互动模型\par
基于罗兰·罗伯特森的全球本土化理论,本研究吸纳了“算法情感转译” 与“本土模式” 等前沿观点,构建了“内核移植—外壳重塑”跨文化传播分析模型(如图4-1所示)。该模型主张,ReelShort的传播策略并非简单的全球化与本土化的二元对立,而是一个动态的剥离与重组过程。\par
“外壳重塑”对应“本土化”维度。这一过程指通过去语境化生产,用符合目标市场审美习惯与文化语境的视听符号剥离、替代具有高认知门槛的中国地缘文化符号,如东方面孔、中式市井场景等,其目的在于最大程度地消解“文化折扣” 。“内核移植”则对应“全球化/普世化”维度。这一过程指保留经过中国市场验证的情感结构(如爽感机制)与伦理逻辑(如善恶有报、阶层逆袭),通过算法情感转译,将其转化为能够引发跨文化受众生理性共鸣的全球通用情感商品 。\par

\begin{figure}[htbp]
\centering
\fbox{\parbox{0.8\linewidth}{\centering 图像占位}}
\caption{图 4-1 国产微短剧跨文化传播的“核-壳”二元互动模型}
\end{figure}
\newpage
\section{“外壳重塑”:视听符号的去语境化生产}
在全球本土化的实践场域中,“外壳”并非仅仅是内容的包装,而是跨文化传播中消融认知壁垒的第一道防线。面对欧美低语境文化的受众,ReelShort采取了一种激进的“去语境化生产”策略。这种策略不再试图向西方解释复杂的中国地缘文化(如“赘婿”的社会学含义或“修仙”的道教背景),而是通过对具身符号、空间符号与题材符号的系统性置换,构建了一个符合西方审美惯习的“拟像世界”,从而实现了从“文化陌生”到“文化亲近”的视觉转译。\par
(一)、具身符号的本土化:从“东方面孔”到“白人精英”\par
影视传播的核心载体是演员的身体。在跨文化传播中,种族特征往往是最先被感知的符号边界。为了规避因种族差异带来的心理距离,ReelShort在选角上彻底摒弃了传统国产剧出海的“华裔面孔”路径,转而启用符合欧美主流审美的本土演员,实现了具身符号的本土化重构。\par
\subsubsection{视觉审美的种族置换}
正如佟亚云(2025)所指出的,欧美受众对男性的审美标准倾向于高鼻梁、深邃眼窝及健硕肌肉线条,这与东亚审美中偏向儒雅、白净的男性形象存在显著差异。\footfullcite{ref05}为了迎合这一期待,ReelShort在选角上大量启用符合“Alpha Male”(阿尔法男性)特征的白人演员——他们通常拥有强壮的体魄和更具侵略性的肢体语言,这直接对应了西方文化中关于“领袖”与“力量”的想象。而在女性角色的选择上,则倾向于金发碧眼或具有邻家气质的白人女性,以契合“灰姑娘”叙事中的无辜感与被救赎感。\par
这种视觉置换在爆款剧《The Double Life of My Billionaire Husband》(亿万富翁丈夫的双重生活)中表现得尤为明显。虽然该剧翻拍自中国微短剧《《闪婚后,傅先生马甲藏不住了》,但男主角Sebastian不再是原版中那个略显隐忍的东亚赘婿形象,而是一位西装革履、气场强大的白人精英。这种形象的重塑,使得“亿万富翁”这一身份符号在西方语境下获得了天然的合法性,消除了观众对“黄皮肤富豪”可能产生的认知失调。\par

\begin{figure}[htbp]
\centering
\fbox{\parbox{0.8\linewidth}{\centering 图像占位}}
\caption{图5-1 ReelShort版男主与国产原版男主形象与形象对比}
\end{figure}
注:左图截取《The Double Life of My Billionaire Husband》男主Sebastian,右图截取中国微短剧《《闪婚后,傅先生马甲藏不住了》男主傅司寒。\par
\subsubsection{表演范式的“错位操演”}
值得注意的是,虽然演员的肤色是西方的,但其表演逻辑却往往被规训于中国微短剧的夸张美学框架内。付超(2026)将其敏锐地概括为“错位操演”:即欧美演员作为表征载体,在执行中国制作团队设定的权力等级逻辑。\footfullcite{ref07}例如,在ReelShort的剧集中,经常出现西方影视中罕见的“扇耳光”、“下跪求饶”或“被泼红酒”等高烈度羞辱戏码。这些肢体语言并非源自欧美日常生活的自然逻辑,而是为了配合“爽感”机制而特意保留的“中国式表演”。这种“西方面孔+中式爆发”的混合杂糅,构成了一种独特的视听奇观,既消除了种族隔阂,又保留了情感宣泄的张力。\par
(二)、空间符号的拟像化:从“在地性”到“无国籍”\par
如果说演员是叙事的主体,那么空间则是叙事的容器。为了使故事能够在全球范围内流通,ReelShort在场景构建上采取了彻底的“去地缘化”策略,剥离了所有可能引发文化折扣的具体地理标志。在ReelShort的剧集中,观众很难看到具有鲜明地域特征的街道、地标建筑或具有特定文化含义的室内陈设。取而代之的,是洛杉矶廉价摄影棚或通用样板房搭建出的“豪宅”、“办公室”与“奢华派对”。这些空间符号具有高度的流动性与通用性——它们既可以是纽约,也可以是伦敦,甚至可以是任何一个西方现代化大都市。\par
这种空间处理方式呼应了岩渊功一提出的“文化嗅觉丧失”策略,即通过抹除产品的民族特征来提升其跨国吸引力。庞华(2025)进一步借用鲍德里亚的“拟像”概念指出,ReelShort构建的是一个属于算法的通用背景。\footfullcite{ref10}在这个拟像空间里,具体的地理位置不再重要,重要的是空间所象征的阶级属性(如巨大的水晶吊灯象征财富,逼仄的地下室象征贫穷)。通过这种空间的标准化生产,ReelShort成功构建了一个全球受众都能无障碍代入的“无国籍”叙事场域。\par

\begin{figure}[htbp]
\centering
\fbox{\parbox{0.8\linewidth}{\centering 图像占位}}
\caption{图 5-2 ReelShort剧中典型的“去语境化”豪宅场景}
\end{figure}

(三)、题材符号的置换:从“修仙等级”到“生物阶层”\par
在“外壳重塑”的过程中,最具挑战性的是如何处理中国网文特有的玄幻与伦理设定。面对“龙王”、“修仙”、“赘婿”等高语境文化符号,ReelShort采取了“同化策略”,将其置换为西方受众熟知的本土神话体系。\par
\subsubsection{权力体系的符号对译}
中国网文中的“修仙等级”(如练气、筑基、元婴)或“宫廷尊卑”(如嫡庶之分),是构建爽感冲突的基础,但这些概念对于西方受众而言晦涩难懂。为此,ReelShort巧妙地引入了西方青少年文学中流行的“ABO”或“狼人(Werewolf)”体系进行对译。以爆款剧《Fated to My Forbidden Alpha》为例,该剧将中国仙侠体系中的等级设定,直接映射为西方狼人题材中的“生物阶层体系”:至高无上的“龙王”被置换为具有绝对统治力的“Alpha”,而受尽欺凌的底层“赘婿”或“庶女”则被对应为“Omega”或被放逐的孤狼。\par
\subsubsection{伦理关系的本土化转码}
除了等级体系,伦理关系也进行了深度转码。中国故事中常见的“包办婚姻”或“家族联姻”,在西方语境下缺乏现实土壤。因此,ReelShort引入了狼人题材中的“Fated Mate”(宿命伴侣)概念。这一设定将原本带有封建色彩的“强制婚姻”,转码为一种不可抗拒的“生物学宿命”。女主角对他人的服从不再是基于儒家伦理的“孝道”或“面子”,而是源于狼人血统中的“Mate Bond”(伴侣联结)。\par
通过这种题材符号的置换,ReelShort不仅保留了“等级森严—遭受压迫—血统觉醒—逆袭打脸”的叙事动线,更赋予了这种逆袭以符合西方语境的合法性解释。这是一种基于市场导向的深度同化,旨在利用目标市场的本土文化符号来降低认知门槛,从而让中国式的故事内核得以在异质文化的躯壳中重生。\par
\newpage
\section{“内核移植”:情感结构的算法转译与能量循环}
在ReelShort的全球化实践中,如果说视听符号的“去语境化”是为了消除陌生感,那么情感结构的“内核移植”则是为了制造成瘾性。这一过程并非简单的故事搬运,而是一种基于算法理性的深度重构。它将中国网络文学中历经市场验证的“爽感”机制,从特定的东亚伦理语境中剥离,转译为一种能够跨越文化边界、直接作用于人类边缘系统的“生理性共鸣”。\footfullcite{ref14}\par
(一)、情感结构的算法转译:从“叙事”到“刺激”\par
传统影视创作遵循的是艺术逻辑,而微短剧遵循的是流量逻辑。ReelShort输出的本质并非“中国故事”,而是一套标准化的“情感刺激技术”。\par
\subsubsection{先抑后扬结构的生理性转码}
中国网文中最核心的“爽感”来源,是“先抑后扬”的叙事节奏。在ReelShort的剧集中,这种节奏被算法精确量化为“黄金前三秒”与“付费点卡点”。以《The Double Life of My Billionaire Husband》为例,剧集开篇并未铺陈复杂的背景,而是直接展示女主角被继母羞辱、被逼替嫁的极端困境。这种高密度的负面刺激在观众大脑中制造了强烈的心理亏空与皮质醇积累。随后,男主角作为亿万富翁的身份在关键时刻(通常是每集的第58秒)瞬间揭露,释放高强度的正向刺激,诱发多巴胺的过量分泌。它不再依赖于观众对“赘婿”这一社会身份的深度文化理解,而是将其转化为一种普世的“受辱—反击”生理反射。无论观众身处何种文化背景,这种基于神经生物学的奖赏预测误差机制都能有效生效。\par
\subsubsection{叙事单元的模块化拼贴}
为了保证这种刺激的稳定性,剧本不再是有机的整体,而是被拆解为“耳光”、“下跪”、“豪车打脸”、“身份亮牌”等标准化的爽点模块。在数据反馈的规训下,编剧被要求在每一集(1.5分钟)内必须完成至少两个反转。这种生产方式实际上是将人类幽微复杂的情感体验,简化为可计算、可复制的数据单元。\footfullcite{ref15}通过这种情感的工业化操作,ReelShort成功将中国式的爽文逻辑,封装为一种全球通用的情感快消品。\par
(二)、伦理内核的普世化:阶层焦虑与回报主义\par
在剥离了具体的地缘文化外壳后,微短剧内核中依然保留了中国通俗文艺独特的伦理底色。这不仅没有成为传播障碍,反而精准击中了全球化断裂带上的普遍社会心理。\par
\subsubsection{从“宗族伦理”到“阶层正义”}
中国故事中的“赘婿”或“庶女”母题,往往植根于东亚特有的宗族尊卑体系。但在ReelShort的转译中,这种伦理冲突被置换为西方语境下更为普遍的“阶层焦虑”与“Underdog(弱者)逆袭”。 在《Fated to My Forbidden Alpha》中,女主角作为“Omega”受到的压迫,被刻画为一种阶级固化下的生存危机。她最终的逆袭,不再是为了光宗耀祖,而是为了争取个体的尊严与生存空间。这种叙事策略巧妙地将东亚的“面子文化”转译为西方的“个人奋斗”话语,使得中国式的“莫欺少年穷”在欧美贫富差距扩大的社会背景下,获得了广泛的共情基础。\footfullcite{ref16}\par
\subsubsection{“回报主义”的伦理补偿}
微短剧的“爽”,本质上是建立在“回报主义”伦理观之上的。与西方影视中常见的道德困境或开放式结局不同,ReelShort的剧集提供的是一种绝对的因果报应:恶毒的继母必然身败名裂,势利的店员必然被解雇,善良的主角必然获得金钱与爱情的双重奖赏。 这种简单粗暴的伦理闭环,龙小农(2024)将其称为“替代性满足”。\footfullcite{ref08}在充满不确定性的现实世界中,这种确定性的正义分配机制,为全球受众提供了一种心理层面的安全感与秩序感。它证明了,即便是在高度个人主义的西方社会,这种基于朴素正义观的伦理内核依然具有强大的穿透力。\par
(三)情感能量的循环:虚拟社群中的互动仪式\par
“内核移植”的最终完成,不仅取决于文本的生产,更取决于受众的接纳。庞华等(2025)引入的“互动仪式链”理论,为我们理解这一过程提供了微观视角。\footfullcite{ref10}基于对TikTok与YouTube评论区的网络民族志观察(如图 5-1 所示),西方受众对ReelShort的反馈已超越了单纯的猎奇,呈现出一种高强度的情感沉浸与社群共振。\par
1. 从“审视”到“沉浸”:无障碍的情感代入\par
不同于以往中国影视出海常面临的“文化审视”或“猎奇目光”,ReelShort的评论区显示出一种惊人的“无障碍代入感”。在《Fated to My Forbidden Alpha》等剧集的评论中,大量受众表现出对剧情的高度投入和对角色的真情实感。如图中评论所示,“Not me actually being invested”(我竟然真的陷进去了)、“I ship Selene and Alexander”(我站这对CP)等高频话语表明,受众已经完全搁置了对“异文化”的防御机制。这种“去陌生化”的沉浸体验,证明了算法情感转译的成功——它成功将“先婚后爱”、“阶层逆袭”等中国式网文逻辑,转化为了一种能够直接引发全球受众共情的通用情感体验,实现了从“看戏”到“入戏”的跨越。\par

\begin{figure}[htbp]
\centering
\fbox{\parbox{0.8\linewidth}{\centering 图像占位}}
\caption{图 6-1 TikTok 评论区观众对剧情高投入相关评论截图}
\end{figure}

2. 匮乏感的释放:虚拟共在与催更仪式\par
在“互动仪式链”的视阈下,评论区不仅是反馈渠道,更是一个情感能量的增值场域。面对每集1.5分钟的碎片化叙事,受众普遍表现出一种强烈的“未完成情结”与“内容饥渴”。 评论中大量的“keep it coming but too short”(继续更,太短了)、“help me watch this”(求资源)以及对剧情走向的热烈讨论,构成了一种集体的“催更仪式”。在这种仪式中,原本分散在世界各地的原子化受众,通过共享同一种“焦灼”与“期待”,形成了一种“虚拟共在”。这种集体性的情感互动,极大地增强了用户对平台的粘性,将瞬间的生理快感转化为了持续的期待视野,从而在社群层面完成了对中国微短剧生产模式的合法性确认。\par

\begin{figure}[htbp]
\centering
\fbox{\parbox{0.8\linewidth}{\centering 图像占位}}
\caption{图 6-2 TikTok 评论区观众对剧集催更相关评论截图}
\end{figure}

在完成大量的原始语料收集之后,本研究利用文本挖掘工具启动同义词识别并生成高频词云图(如图 6-3所示),以更深入地透视海外受众对出海微短剧的真实认知与情感倾向。\par
从词云分布可以看出:首先,“Werewolf”(狼人)、“Alpha”(阿尔法)、“CEO”、“Slap”(扇耳光)等题材符号的高频出现,表明观众的注意力高度集中在特定的类型化人设与强冲突情节上。观众在观看后,往往针对“狼人变身”或“豪门恩怨”等具体爽点在评论区展开深入讨论,而非关注叙事的逻辑性。其次,“Trashy”(烂俗)、“Cringe”(尴尬/脚趾扣地)与“Satisfying”(满足)、“Addicted”(上瘾)、“Good”(棒)等情感描述词在词云中交织出现且占据显著位置。这种强烈的语义反差,生动地揭示了受众的“矛盾性沉浸”心理——即在认知层面判定其为“烂俗内容”,却在生理层面无法抗拒其带来的“爽感”与满足。再者,“Paid”(付费)、“Money”(钱)、“Finish”(看完)、“Episode”(剧集)、“Help”(求助/求资源)等词汇的极高权重,直观地反映了微短剧“卡点付费”机制对受众产生的强迫性吸引力。在短暂的观影结束后,观众的“完播焦虑”被极大地调动,从而在评论区狂热地寻求免费资源或表达对昂贵费用的“痛并快乐着”的抱怨。最后,值得注意的是,词云中几乎未出现关于“Culture”(文化)、“Meaning”(意义)或“Theme”(主题)等指向深层价值的词汇。这说明观众的互动主要停留于表层的感官欢愉与情绪宣泄,缺乏深度的文化思考,这再次印证了微短剧作为一种“全球化情感快消品”的本质特征。\par

\begin{figure}[htbp]
\centering
\fbox{\parbox{0.8\linewidth}{\centering 图像占位}}
\caption{图6-3 TikTok评论区ReelShort观众评论词云图}
\end{figure}

综上所述,ReelShort的“内核移植”并非一种单向的文化输出,而是一场基于技术理性与人性弱点的精准算计。它通过算法将情感结构标准化,通过伦理转码将中国经验普世化,并通过社群互动将瞬间的快感转化为持久的情感能量。正是这种“外壳去语境化”与“内核强情感化”的辩证组合,构成了中国微短剧出海独特的传播机理。\par
\newpage
\section{结语与展望}
(一)、研究结论:“外核—内壳”共生下的全球传播新范式\par
本研究以ReelShort为典型个案,置于全球本土化的理论下,深入剖析了中国微短剧出海的传播策略与内在机理。研究发现,ReelShort在欧美市场的成功并非偶然的流量红利,而是一种基于数字逻辑与情感计算的“精确制导”。其核心机制在于构建了一个“内核移植”与“外壳重塑”辩证统一的二元互动模型。\par
首先,“外壳重塑”是消融文化折扣的前端策略。 面对欧美低语境文化的受众,ReelShort采取了彻底的“去语境化生产”路径。通过启用白人演员、搭建拟像化的美式场景、置换“狼人/吸血鬼”等本土题材符号,平台成功构建了一个符合西方审美惯习的“无国籍”视听空间。这种策略有效地剥离了原有文化产品中高门槛的地缘特征,使异质文化产品获得了进入主流市场的“通行证”。其次,“内核移植”是维系用户粘性的深层逻辑。在西式外壳之下,ReelShort顽强地保留了中国网络文学中历经市场验证的“爽感”机制与伦理内核。通过算法对先抑后扬叙事节奏的标准化转译,以及对“阶层逆袭”、“善恶有报”等普世伦理的精准投喂,微短剧成功激活了跨越文化边界的生理性共鸣。\footfullcite{ref16}这种情感能量在虚拟社群的互动仪式中不断循环增值,最终完成了从瞬间快感到持续消费的转化。\par
综上所述,ReelShort模式的本质,是作为内核的中国通俗文艺生产经验与作为外壳的全球化视听工业标准的一次深度耦合。它证明了在智能传播时代,跨文化传播的关键不在于输出某种固定的文化本质,而在于如何通过技术理性和情感治理,寻找并满足全人类共通的欲望结构。\par
(二)、现实困境:算法时代的“文化空心”与“审美内卷”\par
尽管“内核移植”与“外壳重塑”策略有效降低了文化折扣,但随着模式的规模化复制,其内生的结构性矛盾日益凸显,主要表现为两大困境。\par
\subsubsection{生产机制的异化:陷入“算法理性”的同质化陷阱}
高度的算法导向虽然提升了传播效率,却也带来了内容的“审美降级”。当剧本创作被简化为大数据的排列组合,当人类幽微的情感体验被拆解为标准化的“爽点”刺激单元,微短剧生产面临着被极致的“算法理性”所吞噬的风险。这种对“黄金三秒”和“反转频率”的过度迷信,导致了严重的路径依赖,使得大量作品沦为缺乏精神内核的“数字快消品”。长期沉浸于这种高强度的多巴胺刺激,不仅可能导致受众的审美疲劳,更可能引发关于“数字精神鸦片”的伦理指责。\par
\subsubsection{文化传播的悖论:陷入“伪在地化”的空心化危机}
ReelShort 虽然通过启用欧美演员和置换题材符号实现了视听层面的“本土化”,但这种“去语境化”的拟像生产,在本质上是一种工具理性的胜利,而非文化的真正融合。它虽然消除了认知隔阂,但也切断了深层文化交流的可能性,造成了“只见模式,不见文化”的“文化空心化”现象。这种“伪在地化”策略虽然在短期内能够收割流量,但难以承载具有厚度的中华人文精神,无法实现从“生理共鸣”到“价值认同”的深层跨越。\par
(三)、优化路径:从“流量迎合”到“数字人文治理”\par
面对上述困境,未来的出海实践需超越单纯的流量逻辑,构建一种兼具技术理性与人文关怀的“数字人文治理”框架。本研究提出以下三维度的进阶路径:\par
\subsubsection{技术向度:从“算法规训”迈向“人机协同”的智能增强。}
面对算法霸权可能导致的创造性枯竭,未来的生产模式应确立“算法为人服务”的伦理边界。一方面,应利用AIGC技术在换脸、配音及场景生成上的优势,降低边际成本,提升工业化效能;另一方面,必须在算法推荐系统中引入“文化多样性”与“审美丰富度”的加权指标,以此对抗算法的同质化倾向。通过建立“人机协同”的生产范式,确保创作者的主体性不被数据洪流淹没,使技术真正成为释放文化生产力的工具,而非扼杀差异性的牢笼。\par
\subsubsection{文化向度:从“去地缘”迈向“文化混杂”的再语境化}
针对“文化空心化”问题,霍米·巴巴的“文化混杂性”理论提供了破局思路。未来的微短剧不应止步于制造一个无国籍的“拟像世界”,而应尝试“再语境化”。即在保留西方视听外壳的同时,审慎而巧妙地植入具有中国美学特质的符号,使其与西方叙事在“第三空间”发生化学反应。这种策略旨在推动从单向的“同化”转向双向的“融合”,培育出一种既非纯粹西方、亦非传统中式,而是具有独特辨识度的“混血”文化形态。\par
\subsubsection{伦理向度:从“生理爽感”迈向“价值共生”的情感升维}
虽然“算法情感转译”成功激发了生理性共鸣,但长期依赖感官刺激易导致边际效应递减。因此,内核的移植需从低维度的“爽感机制”向高维度的“价值共识”跃迁。\footfullcite{ref17}这意味着在保留“逆袭”、“复仇”等强情节框架的同时,需注入更具人类命运共同体意识的普世议题,如女性独立、家庭和解、环境保护等。通过将中国文化中“和合共生”、“推己及人”的伦理智慧融入全球通用的叙事模型,实现从单纯的“情绪宣泄”到深层的“情感治愈”的转化,从而增强用户的情感粘性与品牌忠诚度。\par
(四)、战略展望:从“内容分发”到“模式输出”的范式升级\par
综上所述,ReelShort的实践不仅是一次商业上的突围,更为中国数字文化产业的全球化提供了重要的战略启示。它证明了在智能传播时代,跨文化传播的核心竞争力已不再局限于单一的“内容产品”,而在于底层的“生产模式”与“情感算法”。\par
长期以来,中国文化产品的国际传播多采取“成品出口”或“版权售卖”的初级形态,往往面临“文化赤字”。而微短剧出海展示了一种更高维度的可能性:“隐形嵌入”。ReelShort并非在向西方受众强行灌输“中国故事”,而是在用“中国逻辑”讲述“世界故事”。这种策略使得中国文化产业首次具备了定义全球大众流行文化生产标准的能力。这意味着,未来的文化出海应更多关注微观层面的“情感基础设施”建设。通过掌握底层的算法推荐逻辑、用户运营机制以及标准化的内容生产流程,中国数字平台有望在全球媒介生态中占据更具话语权的生态位,实现从全球文化的“参与者”到“规则制定者”的身份跃迁。\par


\newpage
% ====== 参考文献(按文中出现顺序自动排序)======
\printbibliography[title={参考文献}]

\end{document}
